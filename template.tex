华中科技大学光学与电子信息学院

《信号与系统》课程

工程设计问题设计报告



题目:__________________________

分组号:

组 长:

组 员:___________________________________

时 间: XXXX~XXXX

指导教师:


报告日期: 年 月 日


报告撰写说明

按照参考模板的内容和格式撰写报告
理论模型部分须结合本课程知识分析问题、建立模型
程序设计部分应给出设计思路、主要流程图和关键函数的说明;结果分析不能只是简单给出结论,应结合具体问题,对关键参数或算法在不同取值条件下对结果的影响情况进行分析和总结。如果可能,还应进行误差分析

目 录

1 问题描述 2

2 理论模型 2

2.1 原理分析与设计思路 2

2.2 数学模型 2

3 程序设计 2

3.1 编程思路 2

3.2 主要流程图及说明 2

3.3 结果分析 3

4 总结与体会 3

参考文献 3

附录 程序主要代码 4

1 问题描述
在获取图像的过程中,由于光学系统的像差、光学成像的衍射、成像系统的的非线性畸变、记录介质的非线性、成像过程的相对运动、环境随机噪声等影响,……

2 理论模型
2.1 原理分析与设计思路
图像复原试图利用退化图像的某种先验知识来重建或复原被退化的图像,因此图像复原可以看成图像退化的逆过程,是将图像退化的过程加以估计,建立退化的数学模型后,…………。

2.2 数学模型
要设计针对某个特定输入频率的两极点带通滤波器……

系统函数可写为:

3 程序设计
3.1 编程思路
我们首先读取图像的灰度可得1024×1024的矩阵orgImage[n,m],理论上白点的灰度值为…

3.2 主要流程图及说明
图1. xxxx流程图

(流程图要求用word:插入🡪形状🡪流程图绘制)

由状态图进行状态编码。扭循环计数器状态数:N=2*6=12个。如表1所示为扭循环计数器的状态数。

3.3 结果分析
运行结果
在matlab上运行后得到…

结果验证
我们用画图软件打开图片…

4 总结与体会
参考文献
[1] 康华光.数字电子技术基础(第五版)[M].北京:高等教育出版社,2006

[2] 付家才.电子实验与实践[M].北京:高等教育出版社,2004

附录 程序主要代码